\documentclass[a4paper,10pt]{article}
\usepackage[a4paper,vmargin={25mm,30mm},hmargin={25mm,25mm}]{geometry}

\usepackage{url}
\usepackage{epsfig}

\usepackage{listings}
\lstset{ %
language=C,                % choose the language of the code
basicstyle=\ttfamily\fontsize{11}{12}\selectfont,
%numbers=left,                   % where to put the line-numbers
numberstyle=\footnotesize,      % the size of the fonts that are used for the line-numbers
numbersep=5pt,                  % how far the line-numbers are from the code
backgroundcolor=\color{white},  % choose the background color. You must add \usepackage{color}
commentstyle=\color{gray},
showspaces=false,               % show spaces adding particular underscores
showstringspaces=false,         % underline spaces within strings
showtabs=false,                 % show tabs within strings adding particular underscores
%frame=single,           % adds a frame around the code
tabsize=2,          % sets default tabsize to 2 spaces
captionpos=b,           % sets the caption-position to bottom
breaklines=true,        % sets automatic line breaking
breakatwhitespace=false,    % sets if automatic breaks should only happen at whitespace
escapeinside={\%*}{*)}          % if you want to add a comment within your code
}
\usepackage{xcolor}
\usepackage{textcomp}

\setlength{\parskip}{\baselineskip}

%\usepackage{pgfplots}
%\usepackage{tikz} %For dia files; sudo apt-get install pgf
%\usepgfplotslibrary{external} %to compile pgfplots seperately
%\usetikzlibrary{external} %to compile tikz plots separately
%\tikzexternalize[prefix=figs/] %to compile pgfplots separately
%\usetikzlibrary{arrows,calc}
%\usepackage{standalone} %to compile tikz plots separately


\title{Lab 1: Dynamic Memory: Heap Manager}
%\input{author}

\newcommand{\comment}[1]{}

\date{}
\begin{document}
\maketitle

\section{Introduction}
For this lab you implement a basic heap manager.  
The standard C runtime library provides a standard heap manager
through the {\texttt malloc} and {\texttt free} library calls.  For this lab
you implement your own version of this interface.

First you should understand the interface, purpose, and function of
a heap manager.   On any properly installed Unix system you should be
able to type {\it man malloc} to a shell or terminal window, which
will output documentation for the interface.  (If you
are using your own computer, follow the instructions to install a
Unix/C development environment first.)  You can also type  {\it man malloc}
into Google.   You can also learn about heap managers from the
readings on the course web.

You are asked to provide your own implementation of these heap API
operators called \texttt{dmalloc} and \texttt{dfree}. 
%\footnote{The lab names are commonly prefixed with \texttt{d}, which---as you might have guessed---can be 
%read as \texttt{duke/devil}, or \texttt{dynamic} in this context.}. 
From the
perspective of the {\it user program} (the program using your heap manager in any given process), the behavior should
be equivalent to the standard \texttt{malloc} and \texttt{free}.  We
have supplied some code to get you started, including a header file called \texttt{dmm.h}
Please use that header file and that API
in your solution, and do not change it.

%  If you change the header file, your solution
%cannot be evaluated correctly using autograder tools. Learning to rely on abstractions or
%APIs provided by a 3rd party is a necessary skill for implementing any systems project.

The lab is designed to build your understanding of how data is laid out in
memory, and how operating systems manage memory.
Your heap manager should be designed to use memory efficiently: you should understand the issues related to memory fragmentation, some
techniques for minimizing fragmentation and achieving good memory utilization.

As a side benefit, the lab will expose you to system programming in the C
programming language, and manipulating data structures ``underneath''
the type system.   Many students find this to
be quite difficult.   We strongly encourage you to start early and
familiarize yourself with the C and its pitfalls and with the C/Unix development environment, e.g., by playing
around with the C examples on the course web.  You will need to know some
basic Unix command line tools: {\it man, cd, ls, cat, more/less, pwd,
  cp, mv, rm, diff} and an editor of some kind.  Also, debugging will
go much more easily if you use {\it gdb}, at least a little.  See
resources on the course web.




\comment{We strongly encourage you to start early and invest time in understanding the concepts. The number of lines
of code you write for this lab need not be more than few dozens; however, you might be drawing
lots of diagrams to understand the concepts behind pointers and managing
linked lists, and analyzing the tradeoffs between the implementation choices.}

To submit your solution, follow the instructions on the lab webpage.

\section{Dynamic Memory Allocation}

At any given time, the heap consists of some sequence of {\it blocks}.   Each
heap block is a contiguous sequence of bytes.  Every byte in the
heap is part of exactly one heap block.  Each block is either
allocated or free.

The heap blocks are variably sized.  The borders between the blocks shift as blocks
are allocated (with \texttt{dmalloc}) and freed (with
\texttt{dfree}).  In particular, the heap manager splits and coalesces
heap blocks to satisfy the mix of heap requests efficiently.   In
doing this, the heap manager must be careful to track the borders and
the status of each block.  The following subsections discuss the implementation in more detail.

\subsection{Block metadata: headers}

The heap manager places a {\it header} at the start of each block of
the heap space.  A block's header represents some {\it metadata}
information about the block, including the block's size.    In general
``metadata'' is data about data: the header describes the block, but
it is not user data.  The rest of the block is available for use by the user program.
%(the running program that is using the heap manager in this process). 
The user program
does not ``see'' the metadata headers, which are only for the internal use of the heap manager.

The code we provided defines \texttt{metadata\_t} as a data structure template (a \texttt{struct}
type) for the block headers.  The
intent is that each heap block will have a \texttt{metadata\_t}
structure at the top of it, whether the block is allocated or free.
The \texttt{metadata\_t} structure is defined in
\texttt{dmm.c} as:

\begin{lstlisting}
typedef struct metadata {
      /* size contains the size of the data object or the amount 
       * of free bytes 
       */
        size_t size;
        struct metadata* next;
        struct metadata* prev; 
} metadata_t;
\end{lstlisting}

The block header is useful for two reasons. First, a block's header represents whether
the block is allocated or free.  A heap manager must track that
information so that it
does not allocate the same region or overlapping regions of memory for
two different \texttt{dmalloc()} calls by accident. 

Second, the block headers help to track and locate the borders between
heap blocks.   This makes it possible to coalesce free heap blocks to
form larger blocks, which may be needed for later large \texttt{dmalloc}
requests.  The supplied
\texttt{metadata} structure makes it easy to link the headers of the free blocks
into a list (the {\it free list}).   

There are many ways to implement a heap manager.  The most efficient
schemes also place a {\it footer} at the end of each block.   You may
use footers if you wish, but they are not required.   

\subsection{Initialization: The sbrk system call}

When a heap manager initializes it obtains a large slab of virtual
memory to ``carve up'' into heap blocks to store the individual items
or objects.
It does this by requesting virtual memory from the
operating system kernel using a system call, e.g., \texttt{mmap} or
\texttt{sbrk}.   This is described in the various text resources on the
course web.   The supplied
code uses \texttt{sbrk} to extend the data segment of the virtual
address space.  The \texttt{sbrk} system call causes a region of the virtual memory
that was previously unused (and invalid) to be registered for use: the
kernel sets up page tables for the region and arranges that each page
of the region is zero-filled as it is referenced.  Then  \texttt{sbrk}
returns a pointer to the new region (the ``slab'').

Initially the heap consists of a single
free heap block that contains the entire slab.  The supplied code casts
the slab address to a pointer to a \texttt{metadata\_t} struct and
places it in a global pointer variable
called \texttt{heap\_region}:

\begin{lstlisting}
metadata_t* heap_region = (metadata_t*) sbrk(MAX_HEAP_SIZE);
\end{lstlisting} 

Thus the \texttt{heap\_region} pointer references the header for
the first (and only) block in the heap, which is a free block.

A ``real'' heap manager may obtain additional slabs as needed.  For
this lab we limit the number of \texttt{sbrk} calls to
\texttt{HEAP\_SYSCALL\_LIMIT} for evaluation purposes.  The default
value is 1.

\subsection{Heap Manager API: dmalloc and dfree}

All of the heap blocks you allocate with
\texttt{dmalloc()} should come from the one initial slab.
Whenever {\texttt dmalloc} allocates
a block, it returns a pointer to the block for use by the application
program.  The returned pointer should ``skip past'' the block's header, so
that the program does not overwrite the header.  The returned pointer
should be 
aligned on a longword boundary.  Be sure that you understand what this means and why it is
important.


\begin{figure}[!ht]
\centering
\epsfig{file=fig1.eps,width=3.5in}\hspace{2mm}\hspace{1mm}
\caption{The initial state of the heap: a single block at the head of the \texttt{freelist}.  }
\label{fig:freelist}
\end{figure}

The supplied code includes some macros to assist you in {\texttt dmm.h}.
It also makes it easy to keep track of the
available space using  a doubly linked list of headers of
the free heap blocks, called a \texttt{freelist}.
At the start of the program, the \texttt{freelist} is initialized to
contain a single large block, consisting of the entire slab pointed to
by \texttt{heap\_region}.
%, we use the term \texttt{freelist header} to denote the head of the \texttt{freelist},
%which is useful to iterate through the list. 
Figure~\ref{fig:freelist} shows the initial state of the heap, with the head of the freelist.
%along with the metadata.


\comment{
See Bryant and O'Hallaron~\cite{Bryant}, Chapter 9 for more
efficient structure. You are encouraged to improve the metadata structure for 
better efficiency. In addition, you can earn extra credit for implementing all the 
operations in constant time, i.e., O(1).
}

\comment{
Figure~\ref{fig:freelist} shows an allocated block along with its metadata
information (the blue region). The payload denotes the space allocated to the user.

There are two possible choices during  initialization. You can append
prologue and epilogue blocks to the start and the end of the
\texttt{freelist}, or initialize the \texttt{freelist} pointers to
\texttt{NULL}. 

We provide the code for the latter; however using the former
may help you to tackle the corner cases more succinctly.

\begin{lstlisting}
        size_t max_bytes = ALIGN(MAX_HEAP_SIZE);
        freelist = (metadata_t*) sbrk(max_bytes);
        /* Why casting is needed here? i.e., why (void*)-1? */
        if (freelist == (void *)-1)
                return false;
        freelist->next = NULL;
        freelist->prev = NULL;
        freelist->size = max_bytes-METADATA_T_ALIGNED;
\end{lstlisting}
}

\section{Splitting a free heap block}

It is often useful to split a free heap block on a call to 
\texttt{dmalloc}.   The split produces two contiguous free heap blocks of any
desired size, within the address range of the original block before the split.
You must implement a split operation: without an ability to split, the heap
could never contain more than one block!

For a split, we first need to check whether the requested size is
less than space available in the target block.  If so, the most memory-efficient approach is
to allocate a block of the requested size by splitting
it off of the target block, leaving the rest of the target block free.
The first block is returned to the caller and the second
block remains in the freelist.
The
\texttt{metadata} header in both the blocks is updated accordingly to reflect the
sizes after splitting.
Figure~\ref{fig:split} shows the \texttt{freelist} after a split.

\begin{figure}[!ht]
\centering
\epsfig{file=fig2.eps,width=3.5in}\hspace{2mm}\hspace{1mm}
\caption{Structure of the \texttt{freelist} after a split.}
\label{fig:split}
\end{figure}


\section{Freeing space: coalescing}
The program frees an allocated heap block by calling \texttt{dfree()}, passing a pointer
to the block to free.   The heap manager reclaims the space, making it available for use by a
future \texttt{dmalloc}.
One solution is to just insert the block back into the
\texttt{freelist}.

As blocks are allocate and freed, over time you may end up with adjacent blocks
that are both free.  In that case, it is often useful and/or necessary
to 
combine the adjacent blocks into a single contiguous heap block.
This is called {\it
  coalescing}.   
Coalescing makes it possible to reuse freed space as part of a future block of a larger size.
Without coalescing, a program that fills its heap with small blocks could never allocate a large block, even if it
frees all of the heap memory.   We say that a heap is ``fragmented'' if its space is available only in small blocks.



% For example, consider the case when an allocated block is
%freed. The function \texttt{dfree()} must link that block into the
%\texttt{freelist}. \comment{Now, where should the freed block be placed?} See Figure~\ref{fig:free}.

\begin{figure}[!ht]
\centering
\epsfig{file=fig3.eps,width=3.5in}\hspace{2mm}\hspace{1mm}
\caption{\texttt{freelist} after the first block is freed. }
\label{fig:free}
\end{figure}



You have a couple of options to perform coalescing: First, you can coalesce as
you exit the \texttt{dfree()} call. Second, you can periodically or
explicitly invoke the coalesce function when you are no longer able to
find a sufficiently large block to satisfy a call to \texttt{dmalloc}.

One optimization we can perform is to keep the \texttt{freelist} in sorted
order w.r.t addresses so that you can do the coalescing in one pass of the
list. For example, if your coalescing function were to start at the beginning of the
\texttt{freelist} and iterate through the end, at any block, it could look up
its adjacent blocks on the left and the right (``above'' and
``below'').   If free blocks are
contiguous/adjacent, the blocks can be coalesced. 

If we keep the \texttt{freelist} in sorted order, coalescing two blocks is
simple. You add the
space of the second block and its metadata to the space in the first block. In
addition, you need to unlink the second block from
the \texttt{freelist} since it has been absorbed by the first block.
See Figure~\ref{fig:coalesce}.

\begin{figure}[!ht]
\centering
\epsfig{file=fig4.eps,width=3.5in}\hspace{2mm}\hspace{1mm}
\caption{\texttt{freelist} after the first two blocks are coalesced.}
\label{fig:coalesce}
\end{figure}

\comment{
\section{Finding the right fit for the requested bytes}
Finding the right fit is hard since it requires you to predict the pattern of
\emph{future} requests. For example, suppose after $n$ requests all the blocks 
are exactly four words in size. Does this heap suffer from fragmentation? The
answer depends on future requests. Check the reference~\cite{Bryant} for more material
on this topic.

While, there is lot of research done in finding the right fit algorithms, 
we recommend to start with the simplest choice: the first fit algorithm, which
essentially traverses the \texttt{freelist} from the head until it finds the
spot. Once you are done with first fit and you have a working code, it is
easier to implement best fit or next fit algorithms.

}

\section{Getting started}

Fetch the source code files from the course web into a directory, cd into that directory, 
and type ``make''. Read the handout. Modify the code as directed in the handout. Type
``make'' again. Test by running the test programs. (Just type the name of the program preceded by ./.) 
Repeat. You can debug/execute all the test cases by running ``make
debug'' and ``make test''. 

We recommend that you first implement {\texttt dmalloc} with
splitting.   Test it.  Then implement {\texttt dfree} by inserting
freed blocks into a sorted freelist.   Test it, and be sure you can
recycle heap blocks through the free list.  Then add support for coalescing to
reduce fragmentation.  Then consider alternative ways to make your
heap manager more efficient.

\section{Roadblocks and hints}
You are encouraged to post your questions or issues to piazza~\cite{piazza}. 
The instructor or the TAs will post tips that may point you and others in the right direction. 

One general advice is to write code in small steps and understand the
functionality of provided header files, macros, and code completely before
starting the implementation. Use \texttt{assert} and \texttt{debug}
macros aggressively.
If you crash, use {\it gdb} to figure out where.  It's not hard!  (See
the resources page on the course web.)

\section{What to submit}
First, fetch the source code available at~\cite{lab1}. In \texttt{dmm.c},
you implement two functions:
\begin{enumerate}
 \item \texttt{dmalloc()}
 \item \texttt{dfree()}
\end{enumerate}

according to the API in \texttt{dmm.h}. Submit a single
\texttt{dmm.c} file along with a \texttt{README} file documenting the
implementation choices you made (more in section~\ref{sec:grading}), the results, the amount of time you spent on the lab, and
also include your name along with the collaborators involved. Use the websubmit
to submit your code by selecting the course
as ``compsc310'' and then the lab label as ``lab1''.

\comment{
Submission 
instructions will be announced on the course page before the deadline.


You can submit
the code using the \texttt{submit210} from any of the CS departmental
machines.
}

You may implement the code any way you like. In particular, you may either
coalesce when you run out of space or when \texttt{dfree()} is
terminating. It must behave in the same way as \texttt{malloc()} does.

As mentioned earlier, we will use the provided version of
\texttt{dmm.h} so it is important that you do not change this header
file in any way. 

In addition, we have provided supplementary files to get you started:
\texttt{Makefile} contains cases for \texttt{compile}, \texttt{debug}, and
\texttt{test}.  \texttt{test\_basic, test\_coalesce, test\_stress1,
test\_stress2} contain test cases scenarios which can be helpful while
debugging. The stress test cases generate variable sized objects and will
randomly call \texttt{dmalloc} and \texttt{dfree} functions. Note: Passing 
\texttt{test\_basic, test\_coalesce} does not mean your code is functional. 
You should aim to pass \texttt{test\_stress2} with high success rate. 
You may want to know that the final test cases need not the same
cases which were provided to you.

\section{Grading}
\label{sec:grading}
The grading is done on the scale of 100 and you have an opportunity to earn 25 extra points.
\begin{itemize}
 \item Correctness: 50 points
 \item Efficiency:  25 points (improve the efficiency using first-fit/best-fit algorithms using the \texttt{metadata} structure provided in the \texttt{dmm.c} file; 
			      efficiency is measured as a combination of utilization and throughput as provided in~\cite{Bryant}; with the provided metadata structure
			      you should at least see a success rate above 80\%)
 \item Readability: 10 points (comment your code where necessary)
 \item README:	    15 points (Your README should contain evaluation of the choices you made in the design of your heap manager.
			      For example, it should contain about the list of information you store in the \texttt{metadata} structure,
			      how the \texttt{coalesce}, \texttt{free}, and \texttt{split} operations benefited from the \texttt{metadata} structure,
			      the space and time overheads and tradeoffs, the results of your test cases, the amount of time you spend on the lab,
			      your name and any collaborators involved, references to any additional resources used, and your feedback on the lab).
			     
 \item Extra credit: 25 points (With address order sorting, the efficiency of all operations is O(\#free blocks). Improve the efficiency of 
			      \texttt{free} and \texttt{coalescing} operations to a constant time, i.e., O(1). Useful reference~\cite{Bryant})

\end{itemize}

\bibliographystyle{plain}
\bibliography{cps210}

\end{document}
